\documentclass[12pt,a4paper,english]{article}
\usepackage{tikz-cd}
\usepackage[a4paper]{geometry}
\usepackage{ctex}
\usepackage[utf8]{inputenc}
\usepackage[OT2,T1]{fontenc}
\usepackage{babel}
\usepackage{dsfont}
\usepackage{amsmath}
\usepackage{amssymb}
\usepackage{amsthm}
\usepackage{stmaryrd}
\usepackage{color}
\usepackage{array}
\usepackage{hyperref}
\usepackage{graphicx}
\usepackage{mathtools}
\usepackage{natbib}

\geometry{top=3cm,bottom=3cm,left=2.5cm,right=2.5cm}
\setlength\parindent{0pt}
\renewcommand{\baselinestretch}{1.3}

\newcommand\restr[2]{{% we make the whole thing an ordinary symbol
  \left.\kern-\nulldelimiterspace % automatically resize the bar with \right
  #1 % the function
  \vphantom{\big|} % pretend it's a little taller at normal size
  \right|_{#2} % this is the delimiter
  }}
  
% definition of the "structure"
\theoremstyle{plain}
\newtheorem*{thm}{Theorem}
\newtheorem*{lem}{Lemma}
\newtheorem*{prop}{Proposition}
\newtheorem*{coro}{Corollary}
\theoremstyle{definition}
\newtheorem*{defi}{Definition}
\newtheorem*{ex}{Example}
\newtheorem*{rem}{Remark}
\newtheorem*{cla}{Claim}

\title{Prime ideals in sectors}
\date{\today}
\author{Milan Berger-Guesneau, Deng Zhiyuan}


% definition of operators
\DeclareMathOperator{\Hom}{\normalfont{Hom}}
\DeclareMathOperator{\Spec}{\normalfont{Spec}}
\DeclareMathOperator{\Specmax}{\normalfont{Spec}_{max}}
\DeclareMathOperator{\Gal}{\normalfont{Gal}}
\DeclareMathOperator{\Proj}{\normalfont{Proj}}
\DeclareMathOperator{\hgt}{\normalfont{ht}}
\DeclareMathOperator{\coht}{\normalfont{coht}}
\DeclareMathOperator{\Frac}{\normalfont{Frac}}

\begin{document}

\maketitle

The aim of this homework is to prove the following result.

\begin{thm}
Let $I$ be an interval contained in $[0,\pi/2)$. The set
\begin{equation*}
    S(I):=\{(z)\in\Spec(\mathbb{Z}[i])\mid\, \arg(z)\in I\}
\end{equation*}
has an analytic density, equal to $2\mu(I)/\pi$.
\end{thm}
In other words, the angles of the prime elements of $\mathbb{Z}[i]$ are uniformly distributed.

\vspace{0.5cm}

\subsection*{Some preliminaries}

Let $\varphi$ be the characteristic function of $I$, extended to $\mathbb{R}$ by $\pi/2$-periodicity. Because this function is $2\pi$-periodic it has a Fourier expansion $\varphi=\sum_{n\in\mathbb{Z}}c_n e^{in\cdot}$, where 
\begin{equation*}
    c_n:=\frac{1}{2\pi}\int_0^{2\pi}\varphi(t)e^{-int}dt=\frac{2}{\pi}\int_I e^{-int}dt.
\end{equation*}

\begin{lem}
If $n\not\equiv 0\mod 4$, then $c_n=0$.
\end{lem}
\begin{proof}
Using the $\pi/2$-periodicity of $\varphi$, we have for all $x$ real
\begin{equation*}
    \sum_{n\in\mathbb{Z}}c_n e^{inx}=\varphi(x)=\varphi\left(x+\frac{\pi}{2}\right)=\sum_{n\in\mathbb{Z}}i^nc_n e^{inx}
\end{equation*}
and so $c_n=i^nc_n$. 
\end{proof}

Therefore we can write 
\begin{equation}\label{FourierPhi}
    \varphi=\sum_{n\in\mathbb{Z}}d_n e^{i4n\cdot}
\end{equation}
where $d_n:=c_{4n}$.
\vspace{0.5cm}

Let, for $n\in\mathbb{Z}$,
\begin{align*}
    \chi_n:&I_{\mathbb{Q}(i)}\longrightarrow\mathbb{C}\\
    &(z)\mapsto (z/|z|)^{4n}=e^{i4n\arg(z)}
\end{align*}
be the Hecke character of $\mathbb{Q}(i)$ given in the course. It's a trivial character if and only if $n=0$.
\vspace{0.5cm}

Put, for all $z\in\mathbb{Z}[i]$, $\psi(z):=\varphi(\arg z)$. Because of equation \eqref{FourierPhi}, we have for all $z\in\mathbb{Z}[i]$
\begin{equation}\label{EqPsi}
    \psi(z)=\sum_{n\in\mathbb{Z}}d_n e^{i4n \arg(z)}=\sum_{n\in\mathbb{Z}}d_n \chi_n(z).
\end{equation}

\subsection*{The link with $L$-functions}

In order to prove the theorem, we have to show the equivalent
\begin{equation}\label{goal}
    \sum_{(z)\in S(I)}\frac{1}{N(z)^s}\sim \frac{2\mu(I)}{\pi}\log\left(\frac{1}{s-1}\right)
\end{equation}
when $s$ goes to $1$ and $\Re(s)>1$ (all the following equivalents will be of this form).

Using equation \eqref{EqPsi}, we obtain for $\Re(s)>1$
\begin{equation}\label{eqM}
    \sum_{(z)\in S(I)}\frac{1}{N(z)^s}=\sum_{(z)\in \Spec(\mathbb{Z}[i])}\frac{\psi(z)}{N(z)^s}=\sum_{n\in\mathbb{Z}}d_n M(s,\chi_n)
\end{equation}
where
\begin{equation*}
    M(s,\chi_n):=\sum_{(z)\in \Spec(\mathbb{Z}[i])}\frac{\chi_n(z)}{N(z)^s}.
\end{equation*}
On the other hand, we have
\begin{align*}
    \log L(s,\chi_n)&=\sum_{(z)\in \Spec(\mathbb{Z}[i])}\sum_{k\geqslant 1}\frac{\chi_n(z)^k}{kN(z)^{ks}}\\
    &=M(s,\chi_n)+\sum_{(z)\in \Spec(\mathbb{Z}[i])}\sum_{k\geqslant 2}\frac{\chi_n(z)^k}{kN(z)^{ks}}.
\end{align*}
The second term in the right hand side is holomorphic in a neighborhood of $1$. Indeed, the estimate
\begin{align*}
    \sum_{(z)\in \Spec(\mathbb{Z}[i])}\sum_{k\geqslant 2}\left|\frac{\chi_n(z)^k}{kN(z)^{ks}}\right|&\leqslant\frac{1}{2}\sum_{(z)\in \Spec(\mathbb{Z}[i])}\sum_{k\geqslant 0}|N(z)^{-(k+2)s}|\\
    &=\frac{1}{2}\sum_{(z)\in \Spec(\mathbb{Z}[i])}\frac{1}{|N(z)^s|(|N(z)^s|-1)}\\
    &=\frac{1}{2}\sum_{(z)\in \Spec(\mathbb{Z}[i])}\frac{1}{|z|^{2\Re(s)}(|z|^{2\Re(s)}-1)}
\end{align*}
shows that it is a holomorphic function on $\Re(s)>1/2$. It implies
\begin{equation}\label{fctnLM}
    \log L(s,\chi_n)\sim M(s,\chi_n).
\end{equation}

Recall that Hecke $L$-functions associated to non-trivial characters can be extended to entire functions, and that they never vanish at $1$. Hence \eqref{eqM} and \eqref{fctnLM} give
\begin{equation}\label{eqSum}
    \sum_{(z)\in S(I)}\frac{1}{N(z)^s}\sim \sum_{n\neq 0}d_n \log L(1,\chi_n)+d_0\log L(s,\mathds{1})
\end{equation}
where $\mathds{1}=\chi_0$ is the trivial Hecke character. We assume here that the first term in the right hand side of \eqref{eqSum} converges. Because the Hecke $L$-function associated to $\mathds{1}$ has a simple pole at $1$, we can write
\begin{equation*}
    L(s,\mathds{1})\sim \frac{\mu}{s-1}
\end{equation*}
and thus
\begin{equation}\label{eqL}
    \log L(s,\mathds{1})\sim \log\mu+\log\left(\frac{1}{s-1}\right)\sim \log\left(\frac{1}{s-1}\right)
\end{equation}
From $\eqref{eqSum}$ and \eqref{eqL} it follows
\begin{equation*}
    \sum_{(z)\in S(I)}\frac{1}{N(z)^s}\sim d_0\log\left(\frac{1}{s-1}\right)
\end{equation*}
Because $d_0=c_0=2\mu(I)/\pi$, we get the equation \eqref{goal} as wanted. This concludes the proof.
\vspace{0.5cm}

\begin{rem}
We assumed here that the first term in the right hand side of \eqref{eqSum} converges. To solve this problem, one can approximate $\varphi$ by two smooth functions $\varphi_-$ and $\varphi_+$ such that $\varphi_-\leqslant \varphi\leqslant\varphi_+$ and $\|\varphi-\varphi_\pm\|\leqslant\varepsilon$ on a period. Replacing $\varphi$ by $\varphi_\pm$ in what we did before, we get that the first term in the right hand side of \eqref{eqSum} converges. Indeed, because $\varphi_\pm$ is smooth its Fourier coefficients are rapidly decreasing. We therefore obtain the values of two quantities $\delta(I,\varphi_\pm)$ such that $\delta(I,\varphi_-)\leqslant\delta(S(I))\leqslant\delta(I,\varphi_+)$. Letting $\varepsilon$ going to zero, we get the result.
\end{rem}
\vspace{1cm}

The theorem we just proved is a particular case of the following result.

\begin{thm}
Let $K$ be a imaginary quadratic number field of class number $1$, let $a$ be the number of roots of unity in $K$. For every interval $I$ contained in $[0,2\pi/a)$, the set
\begin{equation*}
    S_K(I):=\{(z)\in\Spec(\mathcal{O}_K)\mid\, \arg(z)\in I\}
\end{equation*}
has an analytic density, equal to $a\mu(I)/2\pi$.
\end{thm}

\begin{ex}\textcolor{white}{.}
\begin{itemize}
    \item[1)] For $K=\mathbb{Q}(i)$, we have $a=4$ and so we recover the preceding theorem.
    \item[2)] For $K=\mathbb{Q}(j)$, we have $a=6$ and so the analytic density of $S_{\mathbb{Q}(j)}(I)$ is $3\mu(I)/\pi$.
\end{itemize}
\end{ex}
\vspace{0.5cm}

Let's see how we can adapt the preceding proof. 

As before, let $\varphi$ be the characteristic function of $I$, but this time extended to $\mathbb{R}$ by $2\pi/a$-periodicity. We still have $\varphi=\sum_{n\in\mathbb{Z}}c_n e^{in\cdot}$ with
\begin{equation*}
    c_n=\frac{a}{2\pi}\int_I e^{-int}dt.
\end{equation*}
One see easily that if $n\not\equiv 0\mod a$, then $c_n=0$. Thus we get 
\begin{equation*}
    \varphi=\sum_{n\in\mathbb{Z}}d_n e^{ian\cdot}
\end{equation*}
where $d_n:=c_{an}$. If we put
\begin{align*}
    \chi_n:&I_{K}\longrightarrow\mathbb{C}\\
    &(z)\mapsto (z/|z|)^{an}=e^{ian\arg(z)}
\end{align*}
and $\psi(z)=\varphi(\arg z)$ as before, then equation \eqref{EqPsi} is still valid in this context.

The rest of the proof is pretty much the same. At the end we get
\begin{equation*}
    \sum_{(z)\in S(I)}\frac{1}{N(z)^s}\sim d_0\log\left(\frac{1}{s-1}\right)
\end{equation*}
with $d_0=a\mu(I)/2\pi$.

%\newpage
%\bibliographystyle{unsrt}
%\bibliography{bib.bib}
\end{document}
